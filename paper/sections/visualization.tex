\section{Subspace-Map System}
The prototype system (\autoref{fig:interface}) of Subspace-Map mainly consists of three views: the Map View, the Subspace List View, and the Map Detail View.
% The prototype system (\autoref{fig:interface}) consists of a Toolbar, a Map View, a Subspace List View, a Map Detail View, and two legend panels.

\subsection{Map View}
% Three levels

\begin{figure}[tb]
\centering
\includegraphics[width=1\columnwidth]{interface}
\caption{The user interface of Subspace-Map. (a) Subspace List View shows the dimensions of each subspace sample. Black/White indicates the presence/absence of a certain dimension. (b) Map View shows the distribution of all subspace samples based on their similarities. (c) Map Detail View presents the dimension and data patterns of a selected map region. (d) and (e) explain the visual encoding and map metaphors, and allow users to switch the exploration mode.}
\label{fig:interface}
\end{figure}

The Map View (\autoref{fig:interface}b) presents the hierarchical map as introduced in~\autoref{subsection:visual_encoding}. All the cities are shown as hexagons, whose colors and distances indicate subspace similarity. Clusters and sub-clusters are conceptualized as countries and provinces respectively, with the representatives being their capitals. The outliers become islands and municipalities. Users can navigate through different cluster levels by zooming and panning. At the municipal level, the natural factors of each city are shown in a fan-shape glyph (\autoref{fig:interface}d and \autoref{fig:case1_local}a). Each filled/unfilled fan indicates the presence/absence of a dimension. We use transparency to encode the stability of each dimension (see~\autoref{subsection:data_preprocessing}), which helps to reveal the shared dimension patterns between each subspace and its neighbors (\autoref{fig:case1_local}b). Users can switch to travel mode to trace pattern transitions between subspaces. Two travel modes are available: air-travel and ground-travel. The former supports travel between any city via flight routes, while the latter allows travel between capital cities via land and sea routes (\autoref{fig:case1_ground_travel}).

%The Map View (\autoref{fig:interface}b) shows our constructed map, where each subspace is regarded as a city and visualized as a hexagon. Based on similarity, the subspaces are hierarchically clustered to form sub-clusters and clusters, which are represented as provinces and countries, respectively. The representative subspaces in provinces and countries are defined as provincial capitals and national capitals. Outliers in the first-level and second-level clusterings become islands and municipalities. The color of the subspaces represents similarities between them in another way. Routes are provided for users to travel. The natural factors of a city are presented in a fan-shape glyph, of which legend is shown in \autoref{fig:interface}d. Each dimension corresponds to a fan. The dimensions contained in the subspace are represented by filled fans, and the dimensions not contained are represented by unfilled fans. The transparency indicates the stability of the corresponding dimension. The more stable the dimension, the less transparent it is.

%The above visual elements are organized according to the map level, i.e., national level, provincial level, and municipal level. Users can navigate the map by panning and zooming, and switch levels by double-clicking the region or clicking the button in the legend panel (\autoref{fig:interface}e), just like with an online map. They can also switch to travel mode to trace pattern transitions between subspaces. Two travel modes are available: air-travel and ground-travel. The former supports travel between any city via flight routes, while the latter allows travel between capital cities in a more stable way via land and sea routes (\autoref{fig:case1_ground_travel}).

% The Map View (\autoref{fig:interface}b) shows the distribution of subspaces based on similarities, where each subspace is regarded as a city and visualized as a hexagonal grid. Countries are composed of similar cities and can be further divided into provinces and municipalities based on the finer granularity of similarity. The capital city is chosen to show the main characteristics within the corresponding region. Cities outside the country become islands. The colors of cities represent similarities between them in another way. Routes are provided for users to travel. The natural factors of a city are presented in a fan shape, of which legend is shown in \autoref{fig:interface}d. These visual elements are organized according to map levels, i.e., national level, provincial level, and municipal level. Users can browse the map by panning and zooming, just like using an online map. They can also switch levels by double-clicking the region or clicking the button in the legend panel (\autoref{fig:interface}e).

% Users are able to switch to travel mode to trace pattern transitions between subspaces. Two travel modes are provided: air-travel and ground-travel. The former has a lot of freedom, which supports travel between any pair of cities. However, users may suffer from abrupt visual changes when they travel between two highly different cities. The latter is more targeted. After clicking on two capital cities, users can travel between them along land routes and sea routes to trace pattern changes in a more stable way (\autoref{fig:case1_ground_travel}).

\subsection{Subspace List View}
The Subspace List View (\autoref{fig:interface}a) provides a list of sampled subspaces. A dimension histogram shows the dimension distribution. The black and white buttons below each dimension bar are used to filter the subspaces that contain or do not contain the corresponding dimension. Each row in the list represents a subspace whose color matches that of the corresponding subspace in the Map View. A black/white rectangle at the corresponding dimension position indicates whether the subspace contains the dimension or not.
% The order of the countries is determined by the y-coordinates of their national capitals in the Map View, and the subspaces in each country are ordered according to their traversal order.
The islands are placed at the bottom of the list. Users can select subspaces by clicking. When they switch to ground-travel mode, the list shows all capital cities and cities passed through during the trip.

% The Subspace List View (\autoref{fig:interface}a) provides a list of subspace samples. A dimension histogram indicates the dimension distribution of the currently shown subspaces, in which the length of the bar encodes the frequency of the respective dimension. The black and white buttons below each dimension bar are used to filter subspaces of which the corresponding dimension is active or not, respectively. Each row in the list represents a subspace whose color matches the color of the corresponding subspace in the Map View. The order of the countries is determined by the y-coordinates of their national capitals in the Map View, and the subspaces in each country are ordered according to their traversal order. The islands are placed at the bottom of the list. The black/white rectangle at the corresponding dimension position is used to indicate whether the subspace contains the dimension, of which the encoding method corresponds to that of the button above. Users can select subspaces by clicking. When they switch to the ground-travel mode, the list shows all the capital cities and the cities that they pass during the trip.

\begin{figure*}[tb]
\centering
\includegraphics[width=\linewidth]{workflow}
\caption{Subspace-Map workflow.
% Visual analysis pipeline of Subspace-Map.
After constructing the map based on the input data, an overview showing the clusters of subspaces is provided. Users can hierarchically conduct exploration at cluster level, sub-cluster level, and subspace level. By analyzing various kinds of patterns and pattern transitions, they can gain insight into the dominant dimensions, stable data patterns, etc.}
\label{fig:workflow}
\end{figure*}

\subsection{Map Detail View}
The Map Detail View (\autoref{fig:interface}c) displays the dimension and data patterns of subspaces. MDS projections show the data patterns of representative subspaces, with point opacity encoding data stability. Procrustes transformation~\cite{BorgGroenen2005} is used to avoid abrupt changes between projections, in order to preserve users' mental maps. A stability matrix is provided (\autoref{fig:case1_country A}) to help compare different data patterns. It shows all data items in a matrix in their original order. Users can always switch between the matrix and the projection. Icons on the boundary circle indicate featured dimensions, whose names are specifically displayed. 
At the national/provincial level, this view shows the data stability and featured dimensions of the cluster/sub-cluster members. At the municipal level, it only shows information about the chosen subspace. In ground-travel mode, it shows the origin, the current location, as well as the destination.
%Furthermore, since different projections can vary greatly, it is difficult to compare the stability of the corresponding data items through them alone. To solve this problem, a stability matrix is provided (\autoref{fig:case1_country A}). It places the data items evenly within the inner square of the boundary circle. Users can switch the display type by clicking the button at the top of the view.

% The Map Detail View (\autoref{fig:interface}c) displays the dimension and data patterns of subspaces. We project the subspace to two-dimensional space using MDS, in which the point opacity encodes the data stability. To avoid the arbitrary rotation and flipping between projections of different subspaces, we apply Procrustes transformation~\cite{BorgGroenen2005}, a geometric transformation technique that can be used to find the best overlap between two sets of positions, to minimize the geometric differences between them, thus preserving users' mental map.
% Furthermore, given that projections can vary considerably across subspaces, it is probably difficult to compare the stability of the corresponding data items by providing only projections.
% % Furthermore, considering that the projections of different subspaces can vary greatly, only providing projections probably makes it difficult to compare the stability of corresponding data items.
% To cope with this problem, a stability matrix is also provided (\autoref{fig:case1_country A}). It places the data items evenly within the inscribed square of the boundary circle. Users can switch the display type by clicking the button at the top of the view. Featured dimensions are represented by the icon on the boundary circle. Their names are marked next to the icon and legend is shown in \autoref{fig:interface}d.
% % Furthermore, considering that showing the projection of each subspace makes it difficult to compare the neighboring structures between them, to cope with this problem, a unified representation is also provided. Users can switch the display type by clicking the button at the top of the view.

%In the initial state, this view displays the national capitals representing the countries.
% They are sorted by the y-coordinates in the Map View. This sorting method also applies when users select a country, and the provincial capitals representing the provinces are displayed.
%When users select a country, the provincial capitals representing the provinces are displayed.
%In both cases, data stability and featured dimensions are calculated based on all subspaces in the corresponding region. When users switch to travel mode, the view displays individual subspaces. In this case, data stability is calculated based on that subspace, and icons indicate the dimensions it contains. In ground-travel mode, projections of the origin, current location, and destination are displayed. In air-travel mode, it only displays the projection of the current location.

% In the initial state, this view shows the countries. Their projections are those of the national capitals, sorted by the y-coordinates of the national capitals in the Map View. This sorting method also applies when users select a country, in which case the provinces are shown. When the view shows countries or provinces, the data stability as well as the featured dimensions are computed based on all subspaces within the corresponding region. This view also changes when users switch to the travel mode, in which the data stability is computed based on the displayed subspace and the icons on the boundary circle indicate the contained dimensions of the subspace. Three projections are shown in the ground-travel mode, namely the projections of the origin subspace, the subspace at the current location, and the destination subspace. In the air-travel mode, it shows the projection at the current location.

\subsection{Exploration Workflow}
All three views are highly coupled to support the subspace exploration workflow (\autoref{fig:workflow}). As a start, an overview is given showing multiple clusters of subspaces. Then users can drill down to different cluster levels and conduct their exploration. They can analyze the dimension and data patterns of each cluster/sub-cluster in the Subspace List View and the Map Detail View. It helps to reveal data patterns shared by most cluster members, as well as dominant dimension patterns that may account for it. Following the suggested routes, users can quickly browse the transition of patterns across different cluster regions. At the municipal level, they can further understand local subspace similarities by analyzing the glyph patterns.
%detect local similarities from the municipal level of the map, and browse pattern transitions by means of two provided travel modes. Based on these explorations, they are able to gain insight into the dominant dimensions, stable data patterns, etc.