\section{Conclusion}
We propose a novel approach called Subspace-Map to help users visualize and explore subspaces. It presents subspaces with various map metaphors, such as representing clusters as regions, showing representatives as capital cities, etc. Routes are also built, through which users are able to trace the transition of patterns between subspaces. We develop a prototype system and demonstrate its effectiveness through two case studies with real-world datasets. A comparison with state-of-the-art methods shows its advantages in different aspects. In the future, we will enhance the analytical functions of Subspace-Map and provide more guidance for subspace exploration.