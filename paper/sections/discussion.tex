\section{Discussion}
\label{section:discussion}
In this section, we discuss the current limitations and the possible extensions of Subspace Map.

%Subspace-Map is designed to help users explore subspaces from multiple aspects. We use map metaphors to visualize the abstract exploration space, where subspaces are represented as cities. Based on the data similarity between subspaces, hierarchical clustering is performed to generate different levels of clusters. In order to summarize the data-dimension interplay, we extract the data and dimension patterns for each cluster. We also provide two exploration modes, namely the air-travel and ground-travel modes, to support free and guided subspace explorations respectively.

% \textbf{Computational scalability.} Now consider a dataset with $n$ data items and $d$ dimensions. The calculation cost mainly lies in the dimensionality reduction algorithm and $k$-NN algorithm. For each subspace, the time complexity of computing its projection through MDS is $\mathcal{O}\left(n^3\right)$, and computing its $k$-NN list requires additional $\mathcal{O}\left(d\cdot n\right)$ runtime. As the dimensionality increases, the computational cost will soon become prohibitive. We can mitigate this problem in three aspects. First, we can further reduce the number of subspaces by applying the proposed subspace sampling strategy. Second, the computation for each subspace is mutually independent. We can adopt parallel computing to speed up the process. Last but not least, we only need to run the preprocessing once. Results can be preserved to avoid redundant calculations.

% \textbf{Visual scalability.} When constructing the map, the visual elements in it are organized hierarchically, and the size of the hexagon representing the subspace adaptively changes with the number of subspaces. This makes the map have good scalability. However, we still need to control the number of subspaces to a reasonable scale to reduce the burden of computing and user exploration. For the Map Detail View, when there are a large number of data items, the overlapping on the scatterplot becomes very serious. This issue could be solved by replacing the current scatterplot representation with aggregation techniques such as the heatmap.

\textbf{Sampling strategy.} We ensure the diversity and representativeness of the sampling results by controlling the sampling frequency of dimension combinations and dimensions. We do not use subspace clustering algorithms directly because they can only find subspaces with clusters. However, we do not consider data patterns, which risks missing important patterns in the data. We argue that important patterns possessed by a number of subspaces can be preserved to some extent due to the uniformity of our sampling guarantees. Later we can better solve this issue by introducing techniques for adaptively sampling subspaces.

\textbf{Map layout.} As shown in \autoref{fig:case1_local}(b), the same local pattern appears in split regions. It suggests that subspaces with similar patterns in the same sub-cluster may be separated due to flaws in the current layout algorithm. At the national and provincial level, this problem is avoided by separating the placement of different clusters and sub-clusters. Therefore, one possible solution is to construct lower level clusters before placing individual subspaces.

% \textbf{Layout defects.} As shown in \autoref{fig:case1_local}(b), the same local pattern appears in split regions. It suggests that some neighborhoods may be separated due to flaws in the current layout algorithm. At the national and provincial level, this issue is avoided by separating the placement of different clusters. Therefore, one possible solution is to construct neighborhood clusters before placing the individual subspaces. An alternative solution is to give preference to vacant grids that are close to the settled neighbors.

\textbf{Color encoding.} To show the similarity between subspaces through color, we project the subspace distance matrix to 3D so that each dimension represents a variable in the RGB color space. This is not an optimal solution because when the parameters in the RGB change, the color change is not so in line the human color perception. And without further constraints, some of the colors assigned to representative subspaces may be too light. We could project the distance matrix to 2D and pick colors from a designed 2D colormap based on the projected position of the subspace.

\textbf{Scalability.} The computational cost mainly lies in the dimensionality reduction algorithm and $k$-NN algorithm. Consider a dataset with $n$ data items and $d$ dimensions. For each subspace, the time complexity of the projection computation via MDS is $\mathcal{O}\left(n^3\right)$, and the $k$-NN list computation requires an additional $\mathcal{O}\left(d\cdot n\right)$ runtime. As dimensionality increases, the computational cost will soon become prohibitive. We can alleviate this problem in the following ways. First, we can reduce the number of subspaces to be computed by subspace sampling. Second, the computation for each subspace is mutually independent. A parallel approach can be used to speed up this process. $k$-NN computation can be further accelerated by approximation. Moreover, after one calculation, the results are saved to avoid repeated calculation.
From a visual perspective, the scalability is influenced by the number of displayed subspaces. For optimization, we hierarchically organize the subspaces in the map and adaptively change the size of the hexagon according to their number. For the Map Detail View, the overlap in the scatterplot becomes severe when the number of data items is large. It can be solved by replacing the current representation with aggregation techniques such as the heatmap.

In our future work, we would like to integrate more analytical techniques. For example, we may provide multiple dimension reduction techniques. Users can choose the most suitable one based on their needs. We also plan to augment the guidance mechanism for automatically providing informative exploration directions or even insights to users.
% applied in high-dimensional data visualization into our method, especially data analysis methods in statistics