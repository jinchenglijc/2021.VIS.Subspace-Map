\section{Introduction}
\label{sect:introduction}

% \maketitle

%Paragraph 1:   what is a subspace and why we are interested in its analysis
In a high-dimensional dataset, each data item is characterized by multiple attributes. However, some of the attributes may not be as informative as the others. Worse still, these redundant attributes may bury deep some important patterns such as clusters and correlations. For example, the physical attributes of a person (e.g. weight and height) are not very helpful for the relational analysis between education level and income. The more time we spend studying redundant attributes, the less likely it is to reveal the information of interest. Therefore, experienced analysts usually pick a small dimension subset that is most relevant to the task at hand before going deep into details. The data space formed by partial dimensions is usually called a \highlight{subspace}.

%Paragraph 2:   clarify the notion of "subspace"
Subspaces can be classified into two types: the axis-aligned and the non-axis-aligned. Axes of the former are parallel to the original dimensions, while axes of the latter are weighted combinations of the original dimensions. Subspaces generated by linear dimensionality reduction~\cite{wold1987principal} are usually non-axis-aligned. They excel at preserving certain data features, but seldom preserve semantics of the original dimensions. In this paper, we only focus on the axis-aligned subspaces, since they are easier to use and interpret for users. We refer to them as ``subspaces'' for short.

%Paragraph 3:   what is so difficult about subspace analysis
Subspace analysis could be extremely complicated given \revise{the} challenges it poses to analysts:
\begin{itemize} 
\item \highlight{The exploration space is \revise{too large} due to the curse of dimensionality.} The curse of dimensionality is \revise{a notion referring} to various problems that occur when analyzing data in high-dimensional spaces. In combinatorics, it appears as the combinatorial explosion \revise{problem}. For a $d$-dimensional dataset, there are altogether $2^{d}-d-1$ subspaces with no less than 2 dimensions. Each additional dimension doubles the efforts needed to explore all \revise{possible} combinations. Without \revise{any} mental map, users are easily overwhelmed faced with such a huge exploration space.
%\item It is hard to compare the data under different dimension settings. The absolute distance between data items is not a workable measure because it is affected by the dimensionality, i.e., the distance is naturally large in a subspace with many dimensions. Therefore, a common, accurate and reliable measurement that is irrelevant to the dimensionality is required.
%Without knowing about such complex interplays, it is difficult to find out a proper subspace with the desired data features.%
\item \highlight{The interplay \revise{is too complicated} between dimension and data pattern.} Including or excluding a single dimension may seem like a subtle change, but it can lead to dramatic data changes. Unable to foresee the data changes, analysts may choose dimensions merely based on semantics, which could lead to undesired results.
\item \highlight{There lacks a direction for subspace exploration.} After \revise{exploring} one subspace, analysts often face the same difficulty: which to explore next? They may search for a candidate with the least dimension and data change. However, such a trial-and-error process could be extremely tedious.
\end{itemize}

%Paragraph 4:   what have been done in the past to facilitate subspace analysis and what are the major defects
% To facilitate subspace analysis, numerous subspace clustering algorithms~\cite{DBLP:journals/tkdd/KriegelKZ09, DBLP:journals/pvldb/MullerGAS09} have been proposed, which aim to find subspaces with valuable data clusters. However, such algorithms easily produce redundant results that need to be further organized with the aid of visualization~\cite{DBLP:conf/ieeevast/TatuMFBSSK12, DBLP:conf/ieeevast/JackleHBKS17}.
% % Besides, they provide no guidance for dimension characterization.
% There are also methods designed to guide users in subspace exploration. However, they either rely on inefficient manual planning~\cite{DBLP:journals/tvcg/ElmqvistDF08, DBLP:journals/tvcg/YuanRWG13, DBLP:journals/tvcg/NamM13} or only apply to 2D subspaces~\cite{DBLP:conf/apvis/NhonW14}.

For many high-complexity problems, heuristic algorithms are often adopted to provide feasible solutions. Subspace search is no exception. Various algorithms~\cite{DBLP:journals/tkdd/KriegelKZ09, DBLP:journals/pvldb/MullerGAS09} have been proposed to find out subspaces with valuable data clusters. However, such algorithms easily produce redundant results that need to be further organized with the help of visualization~\cite{DBLP:conf/ieeevast/TatuMFBSSK12, DBLP:conf/ieeevast/JackleHBKS17}. Besides, they provide no guidance for dimension selection, \revise{making it hard for users to manually adjust the subspaces}. There are also methods aiming to guide users in subspace exploration. However, they either rely on inefficient manual planning~\cite{DBLP:journals/tvcg/ElmqvistDF08, DBLP:journals/tvcg/YuanRWG13} or only apply to 2D subspaces~\cite{DBLP:conf/apvis/NhonW14}.

% Paragraph 5:   what inspire this work and what are our goals
To address the above challenges, we aim to achieve three goals:
\begin{itemize}
\item \highlight{G1} Provide an overview of the exploration space to help users understand relationships between subspaces and build up their mental maps.
\item \highlight{G2} Reveal the interplay between dimension and data pattern, thus provide guidance for dimensional decisions.
\item \highlight{G3} \revise{Guide users through the exploration space, so that they can learn in a short time about the representative subspaces and their differences.}
\end{itemize}

%Paragraph 6:   what have been done in this work and what are the contributions
% We propose Subspace-Map, a visualization approach that utilizes map metaphors to present features of subspaces and guide users in the exploration. In Subspace-Map, the exploration space is visualized as a geographic map with each subspace represented as a city. The landscape of the city is the data patterns, which are shaped by its natural factors (contained dimensions). By comparing the landscapes, we are able to cluster similar cities into larger regions like provinces and countries. To describe each cluster, we extract the featured dimensions and data patterns shared by most cluster members. Such a high-level summarization reflects the complex dimension-data interplay. Routes are built to go through various clusters, allowing users to explore different types of subspaces in a gradual way.

We propose Subspace-Map, a visualization approach that utilizes map metaphors to present \revise{an overview} of subspaces and guide users in the exploration. In Subspace-Map, the exploration space is visualized as a geographic map with each subspace conceptualized as a city (\highlight{G1}). The landscape of each city is the data patterns, which are shaped by the natural factors (i.e. dimension combination). By comparing the landscapes, we are able to cluster similar cities into larger regions like provinces and countries. To better describe each cluster, we extract the featured dimensions and data patterns that are shared by most cluster members. Such a high-level summarization \revise{is able to reveal} the complex dimension-data interplay (\highlight{G2}). Routes are built to go through various \revise{cities}, allowing users to explore different types of subspaces in a gradual way (\highlight{G3}).

% Our main contributions are:
% \begin{itemize} 
% \item We introduce a novel map metaphor design for subspaces. The map provides an overview to display the relationship between subspaces
% \item We develop a prototype system
% \end{itemize}

%Paragraph 7:   the overall structure of this paper
The remainder of this paper is structured as follows. In Section 2, we briefly review the literature regarding high-dimensional data visualization and subspace analysis. We derive design considerations and introduce the conceptual design of Subspace-Map in Section 3. Details for map construction are elaborated in Section 4. Section 5 introduces the user interface and interactions of our prototype system.
In Section 6, we evaluate Subspace-Map with \revise{real-world cases and user studies}.
% In Section 6, we present two case studies and a comparison with four relevant state-of-the-art approaches to demonstrate the effectiveness of Subspace-Map.
Section 7 discusses current deficiencies and future work. The final section concludes the whole paper.